%% Generated by Sphinx.
\def\sphinxdocclass{report}
\documentclass[letterpaper,10pt,english]{sphinxmanual}
\ifdefined\pdfpxdimen
   \let\sphinxpxdimen\pdfpxdimen\else\newdimen\sphinxpxdimen
\fi \sphinxpxdimen=.75bp\relax
\ifdefined\pdfimageresolution
    \pdfimageresolution= \numexpr \dimexpr1in\relax/\sphinxpxdimen\relax
\fi
%% let collapsible pdf bookmarks panel have high depth per default
\PassOptionsToPackage{bookmarksdepth=5}{hyperref}

\PassOptionsToPackage{booktabs}{sphinx}
\PassOptionsToPackage{colorrows}{sphinx}

\PassOptionsToPackage{warn}{textcomp}
\usepackage[utf8]{inputenc}
\ifdefined\DeclareUnicodeCharacter
% support both utf8 and utf8x syntaxes
  \ifdefined\DeclareUnicodeCharacterAsOptional
    \def\sphinxDUC#1{\DeclareUnicodeCharacter{"#1}}
  \else
    \let\sphinxDUC\DeclareUnicodeCharacter
  \fi
  \sphinxDUC{00A0}{\nobreakspace}
  \sphinxDUC{2500}{\sphinxunichar{2500}}
  \sphinxDUC{2502}{\sphinxunichar{2502}}
  \sphinxDUC{2514}{\sphinxunichar{2514}}
  \sphinxDUC{251C}{\sphinxunichar{251C}}
  \sphinxDUC{2572}{\textbackslash}
\fi
\usepackage{cmap}
\usepackage[T1]{fontenc}
\usepackage{amsmath,amssymb,amstext}
\usepackage{babel}



\usepackage{tgtermes}
\usepackage{tgheros}
\renewcommand{\ttdefault}{txtt}



\usepackage[Bjarne]{fncychap}
\usepackage{sphinx}

\fvset{fontsize=auto}
\usepackage{geometry}


% Include hyperref last.
\usepackage{hyperref}
% Fix anchor placement for figures with captions.
\usepackage{hypcap}% it must be loaded after hyperref.
% Set up styles of URL: it should be placed after hyperref.
\urlstyle{same}

\addto\captionsenglish{\renewcommand{\contentsname}{Contents:}}

\usepackage{sphinxmessages}
\setcounter{tocdepth}{0}



\title{Turtle Oxford}
\date{Jan 21, 2024}
\release{}
\author{Ioana Vasile}
\newcommand{\sphinxlogo}{\vbox{}}
\renewcommand{\releasename}{}
\makeindex
\begin{document}

\ifdefined\shorthandoff
  \ifnum\catcode`\=\string=\active\shorthandoff{=}\fi
  \ifnum\catcode`\"=\active\shorthandoff{"}\fi
\fi

\pagestyle{empty}
\sphinxmaketitle
\pagestyle{plain}
\sphinxtableofcontents
\pagestyle{normal}
\phantomsection\label{\detokenize{index::doc}}


\sphinxstepscope


\chapter{Usage}
\label{\detokenize{usage:usage}}\label{\detokenize{usage::doc}}

\section{Installing Python}
\label{\detokenize{usage:installing-python}}
\sphinxAtStartPar
A good place to start is to consult the official Python starter documentation here: \sphinxurl{https://wiki.python.org/moin/BeginnersGuide}.

\sphinxAtStartPar
First, you need to have both Python and pip installed. Some operating systems come with the former already pre\sphinxhyphen{}installed. To check if it is indeed installed,
you can run the following command in a shell (this is a terminal window on a Unix machine or a Command Prompt or PowerShell window on a Windows machine):

\begin{sphinxVerbatim}[commandchars=\\\{\}]
\PYG{n}{python} \PYG{o}{\PYGZhy{}}\PYG{o}{\PYGZhy{}}\PYG{n}{version}
\end{sphinxVerbatim}

\sphinxAtStartPar
If the version is too old (starts with 2.X), then you need to follow the installation steps anyway.

\sphinxAtStartPar
To install Python, please go to \sphinxurl{https://www.python.org/downloads/} where you can download a Python installer. If you’re having any issues or questions related to
the installer, you should consult this installation guide which has information for most operating systems: \sphinxurl{https://docs.python.org/3/using/index.html}.


\section{Installing Pip}
\label{\detokenize{usage:installing-pip}}
\sphinxAtStartPar
The official guide to installing Pip can be found here: \sphinxurl{https://pip.pypa.io/en/stable/installation/}.

\sphinxAtStartPar
Pip is a python package manager which you can use to install python libraries or applications.

\sphinxAtStartPar
First, you may already have pip installed, especially if you’ve just downloaded and installed Python. You can check by running the following command in a shell:

\begin{sphinxVerbatim}[commandchars=\\\{\}]
\PYG{n}{pip} \PYG{o}{\PYGZhy{}}\PYG{o}{\PYGZhy{}}\PYG{n}{version}
\end{sphinxVerbatim}

\sphinxAtStartPar
If you do not have it installed, you can install it by running the following commands
\begin{description}
\sphinxlineitem{For Max or Linux::}
\sphinxAtStartPar
python \sphinxhyphen{}m ensurepip \textendash{}upgrade

\sphinxlineitem{For Windows::}
\sphinxAtStartPar
py \sphinxhyphen{}m ensurepip \textendash{}upgrade

\end{description}


\section{Installing an IDE}
\label{\detokenize{usage:installing-an-ide}}
\sphinxAtStartPar
You can, of course, write your code in any text editor you want, then save it with a \sphinxtitleref{.py} extension and run it from a shell like so
\begin{description}
\sphinxlineitem{For Mac or Linux::}
\sphinxAtStartPar
python example.py

\sphinxlineitem{For Windows::}
\sphinxAtStartPar
py example.py

\end{description}

\sphinxAtStartPar
But certain text editors provide a bit more functionality for writing code, such as syntax highlighting, auto\sphinxhyphen{}formatting and a shell within the same window.
A very popular such editor (also called an IDE) is Visual Studio Code. Although this is a Microsoft product, there are free versions of it available for most
operating systems, so you do not need to own a Windows machine to be able to install and use it. You can download an installer from \sphinxurl{https://code.visualstudio.com/}.


\section{Installing Turtle Oxford for Python}
\label{\detokenize{usage:installing-turtle-oxford-for-python}}
\sphinxAtStartPar
The code for Turtle Oxford is hosted entirely on GitHub at the following address: \sphinxurl{https://github.com/turtleoxford/turtle-python}.
\begin{description}
\sphinxlineitem{You can install the package in a shell, using pip with the following command::}
\sphinxAtStartPar
pip install git+https://github.com/turtleoxford/turtle\sphinxhyphen{}python.git

\end{description}


\section{Executing one of the examples}
\label{\detokenize{usage:executing-one-of-the-examples}}
\sphinxAtStartPar
You can execute one of the Oxford Turtle Python examples from \sphinxurl{https://github.com/turtleoxford/turtle-python/tree/main/examples} by downloading them
and then running the following command in a shell (\sphinxtitleref{draw\_pause.py} is just one of the file names, it could be any of them):
\begin{description}
\sphinxlineitem{On Mac or Linux}
\sphinxAtStartPar
python draw\_pause.py

\sphinxlineitem{Or on Windows}
\sphinxAtStartPar
py draw\_pause.py

\end{description}


\section{Writing your own programme}
\label{\detokenize{usage:writing-your-own-programme}}
\sphinxAtStartPar
The structure of one of Oxford Turtle Python programmes is as follows:

\begin{sphinxVerbatim}[commandchars=\\\{\}]
\PYG{k+kn}{from} \PYG{n+nn}{turtle\PYGZus{}oxford} \PYG{k+kn}{import} \PYG{o}{*}  \PYG{c+c1}{\PYGZsh{} This tells the interpreter to import all symbols from the Turtle module}
\PYG{k+kn}{from} \PYG{n+nn}{math} \PYG{k+kn}{import} \PYG{o}{*}           \PYG{c+c1}{\PYGZsh{} Optional: import one or more other packages and their symbols, math is a useful one\PYGZbs{}}

\PYG{k}{with} \PYG{n}{turtle\PYGZus{}canvas}\PYG{p}{(}\PYG{l+m+mi}{0}\PYG{p}{,} \PYG{l+m+mi}{0}\PYG{p}{,} \PYG{l+m+mi}{500}\PYG{p}{,} \PYG{l+m+mi}{500}\PYG{p}{)} \PYG{k}{as} \PYG{n}{t}\PYG{p}{:} \PYG{c+c1}{\PYGZsh{} Create the canvas with the desired dimensions, you can omit the parameters to use the default size}
    \PYG{o}{\PYGZlt{}}\PYG{n}{Turtle} \PYG{n}{Commands}\PYG{o}{\PYGZgt{}}
\end{sphinxVerbatim}

\sphinxAtStartPar
You can execute this like any other python file, like in the previous section.

\sphinxstepscope


\chapter{API}
\label{\detokenize{api:api}}\label{\detokenize{api::doc}}

\begin{savenotes}\sphinxattablestart
\sphinxthistablewithglobalstyle
\sphinxthistablewithnovlinesstyle
\centering
\begin{tabulary}{\linewidth}[t]{\X{1}{2}\X{1}{2}}
\sphinxtoprule
\sphinxtableatstartofbodyhook
\sphinxAtStartPar
{\hyperref[\detokenize{generated/turtle_oxford:module-turtle_oxford}]{\sphinxcrossref{\sphinxcode{\sphinxupquote{turtle\_oxford}}}}}
&
\sphinxAtStartPar
Turtle Oxford \sphinxhyphen{} a python library for the Oxford Turtle System
\\
\sphinxbottomrule
\end{tabulary}
\sphinxtableafterendhook\par
\sphinxattableend\end{savenotes}

\sphinxstepscope


\section{turtle\_oxford}
\label{\detokenize{generated/turtle_oxford:module-turtle_oxford}}\label{\detokenize{generated/turtle_oxford:turtle-oxford}}\label{\detokenize{generated/turtle_oxford::doc}}\index{module@\spxentry{module}!turtle\_oxford@\spxentry{turtle\_oxford}}\index{turtle\_oxford@\spxentry{turtle\_oxford}!module@\spxentry{module}}
\sphinxAtStartPar
Turtle Oxford \sphinxhyphen{} a python library for the Oxford Turtle System
\subsubsection*{Functions}


\begin{savenotes}
\sphinxatlongtablestart
\sphinxthistablewithglobalstyle
\sphinxthistablewithnovlinesstyle
\makeatletter
  \LTleft \@totalleftmargin plus1fill
  \LTright\dimexpr\columnwidth-\@totalleftmargin-\linewidth\relax plus1fill
\makeatother
\begin{longtable}{\X{1}{2}\X{1}{2}}
\sphinxtoprule
\endfirsthead

\multicolumn{2}{c}{\sphinxnorowcolor
    \makebox[0pt]{\sphinxtablecontinued{\tablename\ \thetable{} \textendash{} continued from previous page}}%
}\\
\sphinxtoprule
\endhead

\sphinxbottomrule
\multicolumn{2}{r}{\sphinxnorowcolor
    \makebox[0pt][r]{\sphinxtablecontinued{continues on next page}}%
}\\
\endfoot

\endlastfoot
\sphinxtableatstartofbodyhook

\sphinxAtStartPar
\sphinxcode{\sphinxupquote{angles}}(degrees)
&
\sphinxAtStartPar
Change the number of degrees in a circle.
\\
\sphinxhline
\sphinxAtStartPar
\sphinxcode{\sphinxupquote{antilog}}(a, b, mult)
&
\sphinxAtStartPar

\\
\sphinxhline
\sphinxAtStartPar
\sphinxcode{\sphinxupquote{back}}(distance)
&
\sphinxAtStartPar
Move back.
\\
\sphinxhline
\sphinxAtStartPar
\sphinxcode{\sphinxupquote{blank}}(*args, **kwargs)
&
\sphinxAtStartPar

\\
\sphinxhline
\sphinxAtStartPar
\sphinxcode{\sphinxupquote{blot}}(*args, **kwargs)
&
\sphinxAtStartPar

\\
\sphinxhline
\sphinxAtStartPar
\sphinxcode{\sphinxupquote{box}}(*args, **kwargs)
&
\sphinxAtStartPar

\\
\sphinxhline
\sphinxAtStartPar
\sphinxcode{\sphinxupquote{circle}}(*args, **kwargs)
&
\sphinxAtStartPar

\\
\sphinxhline
\sphinxAtStartPar
\sphinxcode{\sphinxupquote{colour}}(new\_colour)
&
\sphinxAtStartPar
Set the new colour of the turtle.
\\
\sphinxhline
\sphinxAtStartPar
\sphinxcode{\sphinxupquote{colour\_to\_int}}(colour)
&
\sphinxAtStartPar
Convert the colour parameter from any acceptable format to an integer (from 0 to 255).
\\
\sphinxhline
\sphinxAtStartPar
\sphinxcode{\sphinxupquote{colour\_to\_str}}(colour)
&
\sphinxAtStartPar
Convert the colour parameter form any acceptable format to a string.
\\
\sphinxhline
\sphinxAtStartPar
\sphinxcode{\sphinxupquote{delete}}(s, idx, l)
&
\sphinxAtStartPar

\\
\sphinxhline
\sphinxAtStartPar
\sphinxcode{\sphinxupquote{detect}}(key\_sym, timeout)
&
\sphinxAtStartPar

\\
\sphinxhline
\sphinxAtStartPar
\sphinxcode{\sphinxupquote{direction}}(degrees)
&
\sphinxAtStartPar
Turtle changes direction to face this number of degrees.
\\
\sphinxhline
\sphinxAtStartPar
\sphinxcode{\sphinxupquote{display}}(*args, **kwargs)
&
\sphinxAtStartPar

\\
\sphinxhline
\sphinxAtStartPar
\sphinxcode{\sphinxupquote{divmult}}(a, b, c)
&
\sphinxAtStartPar

\\
\sphinxhline
\sphinxAtStartPar
\sphinxcode{\sphinxupquote{draw}}(func)
&
\sphinxAtStartPar
Private.
\\
\sphinxhline
\sphinxAtStartPar
\sphinxcode{\sphinxupquote{drawxy}}(*args, **kwargs)
&
\sphinxAtStartPar

\\
\sphinxhline
\sphinxAtStartPar
\sphinxcode{\sphinxupquote{ellblot}}(*args, **kwargs)
&
\sphinxAtStartPar

\\
\sphinxhline
\sphinxAtStartPar
\sphinxcode{\sphinxupquote{ellipse}}(*args, **kwargs)
&
\sphinxAtStartPar

\\
\sphinxhline
\sphinxAtStartPar
\sphinxcode{\sphinxupquote{forget}}(n)
&
\sphinxAtStartPar
Forget the last n positions of the turtle
\\
\sphinxhline
\sphinxAtStartPar
\sphinxcode{\sphinxupquote{forward}}(distance)
&
\sphinxAtStartPar
Move forward.
\\
\sphinxhline
\sphinxAtStartPar
\sphinxcode{\sphinxupquote{get\_key\_code}}()
&
\sphinxAtStartPar

\\
\sphinxhline
\sphinxAtStartPar
\sphinxcode{\sphinxupquote{get\_key\_sym}}()
&
\sphinxAtStartPar

\\
\sphinxhline
\sphinxAtStartPar
\sphinxcode{\sphinxupquote{halt}}({[}e{]})
&
\sphinxAtStartPar

\\
\sphinxhline
\sphinxAtStartPar
\sphinxcode{\sphinxupquote{home}}()
&
\sphinxAtStartPar
Move the turtle to the center of the canvas
\\
\sphinxhline
\sphinxAtStartPar
\sphinxcode{\sphinxupquote{intdef}}(s, default)
&
\sphinxAtStartPar

\\
\sphinxhline
\sphinxAtStartPar
\sphinxcode{\sphinxupquote{left}}(degrees)
&
\sphinxAtStartPar
Turn left.
\\
\sphinxhline
\sphinxAtStartPar
\sphinxcode{\sphinxupquote{maxint}}()
&
\sphinxAtStartPar

\\
\sphinxhline
\sphinxAtStartPar
\sphinxcode{\sphinxupquote{mixcols}}(col1, col2, prop1, prop2)
&
\sphinxAtStartPar

\\
\sphinxhline
\sphinxAtStartPar
\sphinxcode{\sphinxupquote{move}}(func)
&
\sphinxAtStartPar
Private.
\\
\sphinxhline
\sphinxAtStartPar
\sphinxcode{\sphinxupquote{movexy}}(*args, **kwargs)
&
\sphinxAtStartPar

\\
\sphinxhline
\sphinxAtStartPar
\sphinxcode{\sphinxupquote{new\_turtle}}(arr)
&
\sphinxAtStartPar

\\
\sphinxhline
\sphinxAtStartPar
\sphinxcode{\sphinxupquote{noupdate}}()
&
\sphinxAtStartPar
Refrain from updating the Canvas when executing all subsequent drawing commands, until update() is called.
\\
\sphinxhline
\sphinxAtStartPar
\sphinxcode{\sphinxupquote{old\_turtle}}()
&
\sphinxAtStartPar

\\
\sphinxhline
\sphinxAtStartPar
\sphinxcode{\sphinxupquote{on\_press}}(event)
&
\sphinxAtStartPar

\\
\sphinxhline
\sphinxAtStartPar
\sphinxcode{\sphinxupquote{on\_release}}(event)
&
\sphinxAtStartPar

\\
\sphinxhline
\sphinxAtStartPar
\sphinxcode{\sphinxupquote{pad}}(s, padding, length)
&
\sphinxAtStartPar

\\
\sphinxhline
\sphinxAtStartPar
\sphinxcode{\sphinxupquote{pause}}(duration)
&
\sphinxAtStartPar
Pause \sphinxtitleref{duration} milliseconds.
\\
\sphinxhline
\sphinxAtStartPar
\sphinxcode{\sphinxupquote{pendown}}()
&
\sphinxAtStartPar
Put down the pen, all movement functions now produce drawings.
\\
\sphinxhline
\sphinxAtStartPar
\sphinxcode{\sphinxupquote{penup}}()
&
\sphinxAtStartPar
Pick up the pen, stop drawing.
\\
\sphinxhline
\sphinxAtStartPar
\sphinxcode{\sphinxupquote{pixcol}}(x, y)
&
\sphinxAtStartPar

\\
\sphinxhline
\sphinxAtStartPar
\sphinxcode{\sphinxupquote{pixset}}(*args, **kwargs)
&
\sphinxAtStartPar

\\
\sphinxhline
\sphinxAtStartPar
\sphinxcode{\sphinxupquote{polygon}}(*args, **kwargs)
&
\sphinxAtStartPar

\\
\sphinxhline
\sphinxAtStartPar
\sphinxcode{\sphinxupquote{polyline}}(*args, **kwargs)
&
\sphinxAtStartPar

\\
\sphinxhline
\sphinxAtStartPar
\sphinxcode{\sphinxupquote{qint}}(s, mult, default)
&
\sphinxAtStartPar

\\
\sphinxhline
\sphinxAtStartPar
\sphinxcode{\sphinxupquote{qstr}}(a, b, decplaces)
&
\sphinxAtStartPar

\\
\sphinxhline
\sphinxAtStartPar
\sphinxcode{\sphinxupquote{randcol}}(n)
&
\sphinxAtStartPar

\\
\sphinxhline
\sphinxAtStartPar
\sphinxcode{\sphinxupquote{remember}}()
&
\sphinxAtStartPar
Add the current coordinates to the history of the turtle
\\
\sphinxhline
\sphinxAtStartPar
\sphinxcode{\sphinxupquote{reset}}(key\_sym)
&
\sphinxAtStartPar

\\
\sphinxhline
\sphinxAtStartPar
\sphinxcode{\sphinxupquote{resolution}}(x, y)
&
\sphinxAtStartPar
Set the resolution of the canvas to x by y
\\
\sphinxhline
\sphinxAtStartPar
\sphinxcode{\sphinxupquote{rgb}}(n)
&
\sphinxAtStartPar

\\
\sphinxhline
\sphinxAtStartPar
\sphinxcode{\sphinxupquote{right}}(degrees)
&
\sphinxAtStartPar
Turn right.
\\
\sphinxhline
\sphinxAtStartPar
\sphinxcode{\sphinxupquote{setx}}(*args, **kwargs)
&
\sphinxAtStartPar

\\
\sphinxhline
\sphinxAtStartPar
\sphinxcode{\sphinxupquote{setxy}}(*args, **kwargs)
&
\sphinxAtStartPar

\\
\sphinxhline
\sphinxAtStartPar
\sphinxcode{\sphinxupquote{sety}}(*args, **kwargs)
&
\sphinxAtStartPar

\\
\sphinxhline
\sphinxAtStartPar
\sphinxcode{\sphinxupquote{status}}(key\_sym)
&
\sphinxAtStartPar

\\
\sphinxhline
\sphinxAtStartPar
\sphinxcode{\sphinxupquote{thickness}}(new\_thickness)
&
\sphinxAtStartPar
Set the thickness of the pen
\\
\sphinxhline
\sphinxAtStartPar
\sphinxcode{\sphinxupquote{turnxy}}(x, y)
&
\sphinxAtStartPar
Turn to face the point (x, y) on the canvas.
\\
\sphinxhline
\sphinxAtStartPar
\sphinxcode{\sphinxupquote{turtle\_canvas}}({[}origin\_x, origin\_y, width, ...{]})
&
\sphinxAtStartPar
Context manager that creates a canvas at the start and halts at the end.
\\
\sphinxhline
\sphinxAtStartPar
\sphinxcode{\sphinxupquote{update}}()
&
\sphinxAtStartPar
Update the Canvas, and continue updating with all subsequent drawing commands.
\\
\sphinxbottomrule
\end{longtable}
\sphinxtableafterendhook
\sphinxatlongtableend
\end{savenotes}
\subsubsection*{Classes}


\begin{savenotes}\sphinxattablestart
\sphinxthistablewithglobalstyle
\sphinxthistablewithnovlinesstyle
\centering
\begin{tabulary}{\linewidth}[t]{\X{1}{2}\X{1}{2}}
\sphinxtoprule
\sphinxtableatstartofbodyhook
\sphinxAtStartPar
\sphinxcode{\sphinxupquote{TurtleCanvas}}()
&
\sphinxAtStartPar
Class with mostly static member describing the turtle and the canvas.
\\
\sphinxbottomrule
\end{tabulary}
\sphinxtableafterendhook\par
\sphinxattableend\end{savenotes}


\renewcommand{\indexname}{Python Module Index}
\begin{sphinxtheindex}
\let\bigletter\sphinxstyleindexlettergroup
\bigletter{t}
\item\relax\sphinxstyleindexentry{turtle\_oxford}\sphinxstyleindexpageref{generated/turtle_oxford:\detokenize{module-turtle_oxford}}
\end{sphinxtheindex}

\renewcommand{\indexname}{Index}
\printindex
\end{document}